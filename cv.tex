%git clone https://github.com/liantze/AltaCV.git CV
%http://mirrors.ibiblio.org/CTAN/fonts/fontawesome/doc/fontawesome.pdf
%pdflatex -> biber -> pdflatex -> pdflatex
\documentclass[10pt,a4paper,ragged2e,withhyper]{AltaCV/altacv}
% We need the EN variable here, but we defined it in the cv_en.tex/cv_hu.tex files.
\newcommand{\huen}[2]{\ifEN#2\else#1\fi}
\ifEN
\else
%Magyar nyelvi támogatás (Babel 3.7 vagy későbbi kell!)
\def\magyarOptions{defaults=hu-min}
\usepackage[magyar]{babel}
\usepackage{t1enc}% ékezetes szavak automatikus elválasztásához
\fi

\geometry{left=1.25cm,right=1.25cm,top=1.5cm,bottom=1.5cm,columnsep=1.2cm}
\usepackage{paracol}
\ifxetexorluatex
% If using xelatex or lualatex:
\setmainfont{Roboto Slab}
\setsansfont{Lato}
\renewcommand{\familydefault}{\sfdefault}
\else
% If using pdflatex:
\usepackage[rm]{roboto}
\usepackage[defaultsans]{lato}
% \usepackage{sourcesanspro}
\renewcommand{\familydefault}{\sfdefault}
\fi
% Change the colours if you want to
\definecolor{szajam}{HTML}{764d41}
\definecolor{SlateGrey}{HTML}{191611}
\definecolor{LightGrey}{HTML}{b78270}
\definecolor{DarkPastelRed}{HTML}{253344}%Nagy cimek
\definecolor{PastelRed}{HTML}{191919}%egyetemek diagram nyelvkorok idezet
\definecolor{GoldenEarth}{HTML}{895d43}%vonalak
\colorlet{name}{szajam}
\colorlet{tagline}{PastelRed}
\colorlet{heading}{DarkPastelRed}
\colorlet{headingrule}{GoldenEarth}
\colorlet{subheading}{PastelRed}
\colorlet{accent}{PastelRed}
\colorlet{emphasis}{SlateGrey}
\colorlet{body}{LightGrey}
% Change some fonts, if necessary
\renewcommand{\namefont}{\Huge\rmfamily\bfseries}
\renewcommand{\personalinfofont}{\footnotesize}
\renewcommand{\cvsectionfont}{\LARGE\rmfamily\bfseries}
\renewcommand{\cvsubsectionfont}{\large\bfseries}
% Change the bullets for itemize and rating marker
% for \cvskill if you want to
\renewcommand{\itemmarker}{{\small\textbullet}}
\renewcommand{\ratingmarker}{\faCircle}
%% sample.bib contains your publications
\addbibresource{sample.bib}
\NewInfoField{locationandmailaddress}{\faMapMarker/\faEnvelope}
\NewInfoField{birthday}{\faHeartbeat}
\NewInfoField{languag}{\faLanguage}
\hypersetup{hidelinks}
\begin{document}
	%%%%%%%%%%%%%%%%%%%%%%%%%%%%%%%%%%%%%%%%%%%%%%%%%%%
	\name{dr. Molnár József}
	\tagline{\faBalanceScale Jogász}
	\photoL{3.5cm}{1}	
	\personalinfo{%
		\birthday{1961. 02. 23.}
		\locationandmailaddress{1123 Budapest, Ráth György utca 1/B 3/2}
		\email{molnar.jozsef.61.email@gmail.com}
		\phone{+36 30 211 5288}
		\github{molnar-jozsef-61/cv}
	}
	\makecvheader
	%\AtBeginEnvironment{itemize}{\small}
	\medskip
	\medskip				
	\columnratio{0.6}
	\begin{paracol}{2}
		\cvsection{\huen{szakmai tapasztalat}{professional experience}}
		\cvevent{Osztályvezető}{BKM Nonprofit Zrt.\\
			(Compliance Szabályozás és Folyamatmenedzsment Osztály)}{2021. 09. 01. -- 2022. 02. 09.}{XI. kerület}
		\textbf{A Társaság "compliance és folyamatmenedzsment" osztályának irányítása:}
		\begin{itemize}[leftmargin=.25in]
			\item  A compliance szempontú megfelelőség biztosítása a szabályozások elkészítésével és kiadásával.
			\item A szabályozók jogi kodifikációjának elvégzése és a szabályozások deregulációja.
			\item A folyamatok optimális kialakításának koordinálása -- a folyamatgazdák bevonásával.
			\item A folyamatgazdák oktatása és a folyamatgazda felelősi rendszer működtetése.
		\end{itemize}\\
		\divider
		\cvevent{Compliance szakreferens}{Főtáv Zrt.\\
			 (Compliace Minőségirányítási és Folyamatmenedzsment Osztály)}{2020. 04. 01. -- 2021. 09. 01.}{XI. kerület}		
		\begin{itemize}[leftmargin=.25in]
			\item  A fentieken túl, (a Főtáv Zrt, a BKM jogelődje 2021.09.01-i létrejöttéig) a vállalat üzletmenet-folytonossági (BCP) rendszerének irányítása.
			\item A vonatkozó szabályrendszer felülvizsgálatának és aktualizálásának koordinálása.
		\end{itemize}\\		
		\divider
		\cvevent{Biztonságszervező és jogi előadó}{Főtáv Zrt. (Biztonságvédelmi Főosztály)}{2015. 02. 01. -- 2020. 04. 01.}{XI. kerület}
		\begin{itemize}[leftmargin=.25in]
			\item A biztonságvédelem szabályozása a biztonságpolitika/stratégia alapján.
			\item A Társaság vezetőinek biztonságvédelmi feladatainak ellátásának segítése.
			\item A Társaság szerződéseinek munka-, tűz-, információ- ipar és titokvédelmi szempontú véleményezése.
			\item A hatóságokkal való kapcsolattartás biztonságvédelmi ügyekben.
		\end{itemize}\\
		\medskip		
		\cvsection{\huen{Oktatás és képzés}{Education}}
		\cvevent{Jogász-közigazgatási szakvizsga}{JPTE-ÁJK}{1996 -- 2000}{Pécs}
		\divider			
		\switchcolumn
		\cvsection{\huen{Életfilozófiám}{My Life Philosophy}}
		
		\begin{quote}
			``A legkönnyebben akkor szerezhetünk igazságot magunknak,\\
			 ha magunk is igazságot szolgáltatunk másoknak.''
		\end{quote}
		
		\cvsection{\huen{{\Large Amikre büszke vagyok}}{Most Proud of}}
		
		\cvachievement{\faTrophy}{Az év oktatója 2X}{2002 / 2003}
		\divider\\
		\cvachievement{\faGraduationCap}{Jogi diploma}{Cum laude \href{https://github.com/molnar-jozsef-61/cv/blob/main/degree_certificate_hu.jpg}{\faFileImage}}\\
		\medskip
		\cvsection{\huen{{\Large hobbi és érdeklődés}}{{\Large hobbies and interests}}}
		\wheelchart{1cm}{0.2cm}{%
			1/10em/accent!70/{Fegyvergyűjtés},
			3/8em/accent!40/{Barkácsolás},
			4/8em/accent!60/{Kertészkedés},
			1/8em/accent!30/{Kirándulás},
			4/8em/accent/{Olvasás},
			2/6em/accent!10/{Főzés}
		}
		\cvsection{\huen{kompetenciák}{competences}}
		\cvtag{MS Office}
		\cvtag{Windows}
		\cvtag{Fényképezés}\\
		\divider\smallskip\\
		\huen{
			\cvtag{Jó kommunikációs készség}			\cvtag{Tanulékonyság}
			\cvtag{Pontosság}
			\cvtag{Csapatjátékos}
			\cvtag{Rugalmasság}
			\cvtag{Terhelhetőség}
			\cvtag{Motivált}\\
			\cvtag{Önálló munkára való képesség}}{%
			\cvtag{Good communication skills}
			\cvtag{Teachable}
			\cvtag{Punctual}
			\cvtag{Team player}
			\cvtag{Flexible}\\
			\cvtag{Motivated}
			\cvtag{Autonomous}
		}
		\\
		\divider\smallskip\\
		\cvtag{Büntetőjogi ismeretek}\\
		\cvtag{Oktatói gyakorlatok}\\
		\cvsection{\huen{Nyelvek}{Languages}}
		\cvskill{Francia \\(B2 / katonai-rendészeti szakmai nyelvvel bővített) \href{https://github.com/molnar-jozsef-61/cv/blob/main/language_certificate.jpg}{\faFileImage}}{3}
		\divider
		\medskip		
	\end{paracol}
		\medskip		
		\medskip
	\begin{paracol}{1}
		\cvsection{\huen{korábbi szakmai tapasztalat}{previous professional experience}}
		\cvevent{Megbízási szerződéses jogász}{Hegyvidéki Önkormányzat}{2011. 03. 16. -- 2013. 08. 30.}{XII. kerület}
		Közterület-használatokkal kapcsolatos ügyintézés, hátralék kezelés.\\
		\divider
		\cvevent{Jogász}{CompArt Stúdió Kft.}{2009. 01. 01. -- 2019. 06. 30.}{Budapest}
		Térinformatikai és közmű nyilvántartási rendszerek tervezése önkörmányzatoknak.\\
		\divider
		\cvevent{Köztisztviselő-Vezető főtanácsos, jogász, hatósági osztályvezető}{Esztergom Város Önkormányzatának Polgármesteri Hivatala (Hatósági Osztály)}{2005. 01. 01. -- 2009. 08. 31.}{Esztergom}
		\begin{itemize}[leftmargin=.25in]
			\item Az önkormányzat rendeleteinek megalkotása.
			\item Az okmányiroda, a közigazgatás, az építéshatósági iroda és a közterületfelügyelet vezetése (40-44 fő).
		\end{itemize}\\
		\divider
		\cvevent{Oktató-tanársegéd}{Zrínyi Miklós Nemzetvédelmi Egyetem (Határőr Tanszék)}{1995. 05. 01. -- 2004. 12. 30.}{Szentendre/Budapest}
		\begin{itemize}[leftmargin=.25in]
			\item Új tantárgyprogramok kidolgozása (büntetőjog, büntető-eljárás jog, kriminológia, bevezetés az
			állam és jogtudományi ismeretekbe, jogtörténelem) valamint óraadás.
			\item A bűnügyi-felderítő szakcsoport vezetője 2000-től.
			\item A Közterületfelügyelet létrehozásának elősegítése a II. kerületi Önkormányzatnál a 2002-2003-as években.
			\item A közterületfelügyelők kiválasztása, és oktatása (sikeresen vizsgázott az összes tanítványom).
			\item A felügyelet működését elősegítő módszertani segédanyagok kidolgozása, és a szabálysértési rendeletek megalkotása.
			\item A bűnügyi ügyek feldolgozásában való közreműködés, Esztergom Város Önkormányzatánál, külső szakértőként.
		\end{itemize}\\
		\divider
		\cvevent{Főosztályvezető-helyettes, határőr őrnagy}{Határőrség Országos Parancsnokság (Fegyelmi Főosztálya)}{1993. 05. 01. -- 1995. 05. 01.}{Budapest}
		\begin{itemize}[leftmargin=.25in]
			\item Fegyelmi vizsgálat, és eljárás lefolytatása a Határőrség személyi állományába tartozó különböző rendfokozatú, beosztású és munkakörben tevékenykedő személlyel szemben. 
			\item A Határőrség fegyelmi helyzetét érintő ellenőrzések, elemzések végzése.
			\item A Határőrségre vonatkozó, a fegyelmi helyzetet érintő vagy azt befolyásoló, jogi és egyéb szakanyagok
			véleményezése.
		\end{itemize}\\
		\divider
		\cvevent{Rendőr alhadnagy, nyomozó, csoportvezető-rendőr alhadnagy}{BRFK. (Életvédelmi és Rablási Alosztálya)}{1984. 03. 16. -- 1993. 05. 01.}{Budapest}
		\begin{itemize}[leftmargin=.25in]
			\item Az egység fő profilja az erőszakos bűncselekmények nyomozása volt, különös tekintettel az emberölés,
			rablás, erőszakos közösülés, zsarolás bűnelkövetésekre.
			\item Rendőr nyomozóként elmélyült ismeretek megszerzése, mind a nyílt, mind az operatív szakterületeken. 
			\item 1993-ban százados, kiemelt főnyomozó és csoportvezető helyettes beosztás elérése.
			\item Nyomozásvezetői feladatok ellátása a	pénzintézetek illetve benzintöltő állomások kárára és az ott dolgozók sérelmére elkövetett fegyveres,vagy felfegyverezve elkövetett rablások nyomozása során.
		\end{itemize}\\
		\divider
		\cvevent{Rendőr hadnagy-bűnügyi szervező}{Rendőrtiszti Főiskola (Bűnügyi Szak)}{1984. 09. 01. -- 1989. 06. 01.}{Budapest}
		Gyilkossági nyomozó, főnyomozó, kiemelt főnyomozó, és csoportvezető helyettesi tevékenységekkel összefüggő feladatok.\\
		\divider		
		\cvevent{Rendőr alhadnagy}{BM Tartalékos Tisztképző Iskola (Rendőr Szak)}{1981. 09. 01. -- 1983. 06. 01.}{Kerepestarcsa}
		Beosztott nyomozói tevékenység.\\
		\divider		
	\end{paracol}		
\end{document}

 
